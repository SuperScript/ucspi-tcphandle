\documentclass{book}
\usepackage{sstdef}

\title{Compiling}
\begin{document}

\section{Compiling a \cmd{tcphandle.c} server}

The \cmd{tcphandle.c} program provides a framework for compiling
preforking
\href{\cmd{tcpserver}}{http://cr.yp.to/ucspi-tcp/tcpserver.html}-like
servers.

Each tcphandle server accepts all \cmd{tcpserver} options, and in
addition:
\begin{itemize}
  \item \cmd{-f \var{lockfile}}:
    Lock \cvar{lockfile} around calls to \cmd{accept}.
\end{itemize}
Utilize this option if your \cmd{accept} system call suffers from the
thundering herd problem.

The concurrency option (\cmd{-c}) sets the number of children that a
tcphandle server forks upon startup.  The default setting is~40.  Each
child process listens for requests on the socket inherited from the
parent process.

If any of its child processes exits, a tcphandle server forks a new
child process to replace the original.  When it receives SIGTERM, a
tcphandle server sends SIGTERM to each of its children, waits for them
to exit, and then itself exits.

Before handling an accepted request, a tcphandle server sets certain
environment variables, a la \cmd{tcpserver}.

\subsection{Server code}
A tcphandle server invokes a \cmd{server} subroutine for each request.
The subroutine reads from the network on file descriptor~0 and writes to
the network on file descriptor~1:
\begin{code}%
  void server(int argc,char * const *argv)
\end{code}
The arguments to \cmd{server} are the command line arguments to the
tcphandle server that remain after option parsing.

The \cmd{server} subroutine is called within a loop, with one iteration
per request.  It must release any resources allocated to handle a
particular request before returning to its caller.  This includes
changes to the environment.

\subsection{Compiling a new tcphandle server}
Create a file \cmd{src/myserver.c} that implements the \cmd{server}
subroutine.  Create a control file \cmd{src/myserver=x} listing all
necessary dependencies.  Then
\begin{code}%
  build myserver
\end{code}

The ucspi-tcphandle package includes two example servers:
\cmd{tcpprint}, and \href{\cmd{tcpperl}}{tcpperl.html}.

\end{document}

